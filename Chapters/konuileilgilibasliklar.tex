\chapter{Konuyla İlgili Çalışmalar}

Giriş ve Motivasyon: Uzaktan algılama verilerini kullanan hava sahnesi sınıflandırması, verilerin özelliklerinden dolayı (az sayıda etiketli veri bulunması ve yüksek boyutluluk) hem askeri hem de sivil alanlarda en zorlu araştırma alanlarından biridir \cite{li2018deep}.Hiperspektral görüntü küpleri, farklı elektromanyetik spektrumlardan (spektral bantlardan) alınan yüzlerce veya binlerce uzamsal görüntüden oluşur. Bu nedenle, araştırma yapanlara aynı anda hem uzamsal hem de spektral bilgi sunarlar. Hiperpektral görüntülerin spektral çözünürlüğü yüksektir. Çünkü elektromanyetik spektrumdan (10-20 nm) dar bantlarda görüntüler almaktadır. Hiperspektral görüntülemede teknolojinin gelişmesiyle birçok ülke bu alana odaklanmıştır. Örneğin, Almanya’nın Environmental Mapping and Analysis Programı (EnMAP), Dünya'nın çevresini küresel ölçekte izlemeyi ve nitelendirmeyi hedeflemektedir. Bu durum günden güne uzaktan algılama görüntülerindeki artışa paralel olarak hiperspektral veri ambarlarında artışa neden olmaktadır. Bunun sonucunda da, görüntülerdeki gizli bilgilerin açığa çıkarılabilmesine imkân tanınmaktadır. Ancak, bu bilgilerin nasıl kullanılacağı daha fazla araştırma gerektiren açık bir konudur. Son yıllarda, birçok gerçek dünya örüntü tanıma uygulaması -özellikle yer bilimi ve uzaktan algılama alanında-, klasik makine öğrenme araçlarına kıyasla (ham veriden etkin özellik çıkartabilme yeteneğinden kaynaklanan) üstün performansı nedeniyle derin öğrenme tekniklerine büyük ilgi göstermektedir. “Derin öğrenme, çok sayıda doğrusal olmayan dönüşümlerden oluşan mimarileri kullanarak verilerde üst düzey soyutlamaları modelleyen bir dizi makine öğrenme algoritmasıdır”\cite{yan2018application}.Adından da anlaşılacağı gibi, Derin Sinir ağları hiyerarşik olarak düzenlenmiş birden fazla gizli katmana sahiptir. Ön katmanlarda basit bilgiler (kenarlar gibi) çıkarılmakta ve sonraki seviyelere iletilmektedir. Bu şekilde, sonraki seviyeler orta seviye bilgileri çıkarmaktadır. Bu işlem (bazı bilgileri girdi olarak alıp çıkış katmanında daha karmaşık bilgileri çıkartma) ağ hiyerarşisi sonuna kadar tekrar eder.\\
\newpage
	Özellikle, Evrişimli Sinir Ağları (Convolutional Neural Networks = CNN) \cite{lecun1998gradient} birçok bilgisayarlı görme problemi için güçlü bir araçtır. Diğer derin öğrenme algoritmalarından farklı olarak, CNN'ler, katlama – evrişim- (filtreleme) işleminin görüntünün belirli bir alıcı alanı (receptive field) üzerinde gerçekleştirildiği evrişim katmanlarını içerir. Derin Öğrenme mimarileri uzaktan algılamada birçok amaç için kullanılabilir: görüntü ön işleme, piksel tabanlı sınıflandırma, hedef tanıma, anlamsal özellik çıkarma ve sahne anlamlandırma. Derin Öğrenmenin uzaktan algılama verilerinde kullanımı bazı nedenlerden dolayı ekstra motivasyona sahiptir: 1) Uzaktan algılama verileri (özellikle çok bantlı ve hiperspektral görüntüler), çoklu spektral bantlar içerir. Bu, birkaç görüntüde bile veri miktarının çok büyük olduğu anlamına gelir. Bu yüzden daha fazla nörona ve daha derin sinir ağlarına ihtiyaç duymaktadır \cite{chen2013aircraft}. 2) Uzaktan algılama görüntüleri, doğal sahne görüntülerinden daha karmaşıktır. Farklı renk, konum, boyut ve yönelime sahip çeşitli nesne türlerinden oluşabilir. Bu karmaşıklık, doğal (natural) görüntü tanıma için ortak bir yaklaşım olan transfer öğrenmenin başarılı bir şekilde uygulanmasına engel olur. Transfer öğrenmede, derin öğrenme modeli çok sayıda etiketli numuneye sahip olan bir veri seti ile (ImageNet gibi) ön eğitimden geçirilmiştir. Ardından, sınırlı eğitim numuneleriyle sadece modelin son 2 veya 3 tam bağlı katmanını yeniden eğitilerek (önceki katmanlar değil) model parametreleri (ağırlıklar) güncellenir. Ayrıca, görsel farklılıklara 4 neden olan farklı sensörlerle görüntüler çekilebilir. 3) Her ne kadar derin öğrenme yöntemleri çok sayıda etiketlenmiş veri ile mükemmel bir performans sergilese de, uzaktan algılamada yalnızca sınırlı sayıda etiketli veri bulunmaktadır. Bu durum derin öğrenme yöntemlerinin performansını sınırlamaktadır. Yukarıda belirtilen nedenlerden ötürü, hiperspektral uzaktan algılama için en uygun derin sinir ağı mimarisinin geliştirilmesi zorlayıcı ve popüler bir araştırma alanıdır.\\
	
	Son zamanlarda, derin öğrenmenin doğal görüntü işleme alanındaki başarısını takiben, bu yöntemler HSI (Hyper-spectral Imaging = Hiperspektral görüntüleme) sınıflamasına da uygulanmış ve etkileyici sonuçlar elde edilmiştir \cite{chen2016deep} \cite{chen2015spectral} \cite{samaniego2008supervised}. Bu yaklaşımlar sırasıyla spektral bilgiden, uzamsal bilgiden ve spektral-uzamsal bilgiden faydalananlar olarak üç ana kategoride ele alınabilir. Uzamsal çözünürlük ile karşılaştırıldığında, spektral çözünürlük nispeten daha yüksektir. Her bir pikselden bir spektral vektör çıkarılabilir ve bu uzamsal piksele gömülü bilgi içeriğini tanımlamak için kullanılır. Geleneksel HSI sınıflandırma yaklaşımları yalnızca spektral bilgileri kullanır. Tipik sınıflandırıcılar arasında k-en yakın komşuluk \cite{samaniego2008supervised}, uzaklık ölçütü \cite{du2001linear}, lojistik regresyon \cite{li2010semisupervised}, ve en büyük olabilirlik kriteri \cite{ediriwickrema1997hierarchical} temelinde uygulanmış olanlar yer almaktadır. Her pikseli doğrudan spektral vektör üzerinden sınıflandırmak çoğu zaman makul ve verimli değildir. \\
	
	Literatürdeki geleneksel HSI sınıflandırma yaklaşımlarında, sınıflandırmayı desteklemek için özellik çıkarma yöntemleri de uygulanmıştır. Uzamsal özellikler genellikle tek bantlı bir görüntüden elde edilir. Günümüzde, uzamsal özellikler yaygın olarak 2D görüntü tarzında çalışan geleneksel Evrişimli Sinir Ağları (örneğin AlexNet \cite{NIPS2012_c399862d}, GoogLeNet \cite{szegedy2015going}, ResNet \cite{he2016deep} ile çıkarılmaktadır. Bununla birlikte, hiperspektral  görüntülerde kullanılan bantların sayısı çok yüksek olduğundan, işlemci teknolojisi gelişmiş olsa bile, bir bütün olarak 2D Evrişimli Sinir Ağlarına (CNN) bir girdi olarak verilemez. Bu nedenle, genellikle minimal 1D CNN mimarileri (evrişim katmanı + havuz katmanı + tam bağlı katman) \cite{hu2015deep} spektral özellik çıkarımı için kullanılır. Ayrıca, hesaplama maliyetini azaltmak ve ağ eğitimini geliştirmek için literatürde dropout ve batch normalization \cite{xu2014regression} teknikleri kullanılmaktadır. Giriş görüntülerini 1D vektörler olarak alan danışmansız derin öğrenme yaklaşımları (örneğin stacked auto encoders ve deep belief networks gibi) aynı zamanda temsil yeteneği olan spektral özellikleri ortaya çıkarmak için de kullanılır. Uzamsal ve spektral özelliklerin 2D ve 1D derin öğrenme mimarilerinden ayrı olarak çıkarılması, son işlem olarak bir birleştirme (fusion) stratejisi gerektirir.
	Bu doğrultuda ilk çalışma \cite{chen2014deep}, yazarların derin spektral özelliği çıkarmak için bir SAE (stacked auto encoder) kullandığı çalışmadır. Bu orijinal çalışmanın ardından, SAE yerine DBN (Deep Belief Network) kullanımı bildirilmiştir \cite{chen2015spectral}. Benzer şekilde, \cite{ma2016hyperspectral} nolu çalışma etkin özelliği öğrenmek ve ince ayar işleminde önsel göreceli bir mesafe eklemek için SAE'yi kullanmıştır.\\
	
Böylece yeterli sayıda etiketli örnek olmadığında istenen özelliklere ilişkin daha etkin bir
yönlendirme sağlanmıştır. \cite{xing2016stacked}, güçlü spektral özellikleri ayıklamak ve sınıflandırma işlemini tamamlamak için stacked denoising auto encoder ağını \cite{vincent2010stacked} kullanmıştır. Benzer bir fikir \cite{he2016hyperspectral} nolu çalışmada benimsenmiştir. Burada, HSI, deep stacking network (DSN) adı verilen yeni bir model ile sınıflandırılmıştır. Bir DSN modeli, her biri bir giriş katmanı, gizli 5 katman ve bir çıkış katmanı içeren birçok basit modülü içerir. Burada giriş katmanından gizli katmana ağırlıklar rasgele veya kontrastlı sapmalarla \cite{hinton2002training}, gizli katmandan çıkış katmanına ise yalancı ters alma işlemi ile başlatımlanır. Zhong vd. \cite{zhong2016diversified} HSI'daki sınıflandırma verimliliğini artırmaya yardımcı olan DBN'nin ön eğitim ve ince ayar işlemlerinde eğitim hedefinin optimizasyonuna çeşitliliği teşvik edici koşulları dâhil etmiştir.\\

	Yukarıda açıklanan spektral özellik çıkartma yöntemleri sadece spektral bilgileri kullanır ve uzamsal bilgileri kullanmaz. Oysa uzamsal bilgi, görüntüdeki komşu pikseller arasında ani bir değişiklik olmadığı hipotezine dayanmaktadır. Bu amaçla, ön işleme olarak, spektral bantları azaltmak için Temel Bileşen Analizi tekniği kullanılır. Daha sonra, derin spektral / uzamsal özellikler elde etmek için tüm hiperspektral görüntü küpüne 2D CNN uygulanır \cite{liang2016hyperspectral}.\\
	
Diğer bir yeni çalışma \cite{li2017hyperspectral} nolu çalışmada önerilmiştir. Burada, uzamsal özellikleri genişletmek için, derin CNN kullanımına dayalı HSI yeniden yapılandırma modeli önerilmiştir. Benzer fikirle, \cite{chen2014deep} ve \cite{lin2013spectral} nolu çalışmalarda PCA, HSI görüntülerinin boyutsallığını azaltmak için kullanılır. Ortaya çıkan veri küpleri tek boyuta indirgenir veya komşu bölgelerden çıkarılır. Bunu daha sonra, sınıflandırma işlemini gerçekleştiren bir SAE izler. \\

	Temel derin öğrenme modelleri (SAE, CNN ve DBN) dışında, son zamanlarda sıralı verilerden yararlanan Recurrent Neural Networks (RNN) de spektral bantların sürekliliğini modellemek için hiperspektral görüntü sınıflandırmasında kullanılmaktadır.
	Piksel tabanlı sınıflandırma yaklaşımlarından ayrı olarak, bir başka bakış açısı da bütün görüntüyü yerel görüntü patch’lerine bölmek ve her bir patch’i önceden tanımlanmış semantik
etiketlerden birine (endüstriyel alan veya yerleşim alanı gibi) atamaktır. Bu top-down yaklaşım genellikle sahne sınıflandırması olarak adlandırılır. Günümüzde bu sınıflandırma yaklaşımı, derin öğrenme mimarileri ile birlikte kullanılmaktadır. 
