\chapter*{ \normalfont Master Thesis}

\begin{center}
    \vspace{-1.25cm}
    SUMMARY\\
    \vspace{0.33cm}
    A HYBRID APPROACH TO THE CLASSIFICATION OF HYPERSPECTRAL SATELLITE IMAGES VAE AND CNN \\
    \vspace{0.33cm}
    Murat Can VARER\\
    \vspace{0.33cm}
    \setstretch{1}
    Karadeniz Technical University\\
    The Graduate School of Natural and Applied Sciences\\
    Computer Engineering Graduate Program\\ 
    Supervisor: Asst. Prof. Murat AYKUT
\end{center}
\vspace{0.5cm}
\setstretch{1.5}
\hspace{0.8cm} In today's technology, where high calculations are made faster, projects with high processing load can be done more easily. Convolutional neural networks (ESA) (Convolution Neural Networks (CNN)) is a Deep learning model developed to solve classification problems for images.
Deep learning refers to the branch of machine learning that represents levels of learning, so the word deep referred to here derives the properties of the images to be classified from the depth of the neural network structure.The Variational AutoEncoder (VAE) is basically a method used as unsolicited learning. The aim is to increase the variety by applying Gaussian noise in the area called the hidden layer in order to obtain more different images by reducing the size of the images.

Remote sensing is the science of obtaining information about the earth's surface. Nowadays, remote sensing plays an important role in determining the gathering areas during the earthquake preparation phase, especially in military airspace. In this thesis, Hyperspectral Images (HSI) taken from satellite will be classified using CNN model.
\\
\\
\textbf{Key Words}: Convolution Neural Networks, Variational AutoEncoder, Deep Learning,


                    \hspace{1.12cm} Remote Sensing

\addcontentsline{toc}{chapter}{SUMMARY}
