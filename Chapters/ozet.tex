\chapter*{\normalfont Yüksek Lisans Tezi}


\begin{center}
\vspace{-1.2cm}
    ÖZET\\
    \vspace{0.33cm}
    HİPERSPEKTRAL UYDU GÖRÜNTÜLERİNİN SINIFLANDIRILMASINDA\\ VAE VE CNN İLE HİBRİT BİR YAKLAŞIM\\
    \vspace{0.33cm}
    Murat Can VARER\\
    \vspace{0.33cm}
    \setstretch{1}
    Karadeniz Teknik Üniversitesi\\
    Fen Bilimleri Enstitüsi\\
    Bilgisayar Mühendisliği Anabilim Dalı\\
    Danışman: Dr. Öğr. Üyesi Murat AYKUT
\end{center}
\vspace{0.5cm}
\setstretch{1.5}
\hspace{0.8cm} Yüksek hesaplamaların daha hızlı yapıldığı günümüz teknolojisinde artık işlem yükü fazla olan projeler daha kolaylıkla yapılmaktadır. Evrişimsel sinir ağları (ESA) (Convolution Neural Networks (CNN) ) görüntüler için sınıflandırma problemlerini çözmek amaçlı geliştirilmiş bir Derin öğrenme modelidir.
Derin öğrenme, öğrenme düzeylerini temsil eden makine öğrenmesinin dalını ifade eder yani burada bahsi geçen derin kelimesi, sınıflandırılacak görüntülerin özelliklerini sinir ağı yapısının derinliğinden alır. Varyasyonel Oto-Kodlayıcı (VOK) (Variational AutoEncoder (VAE) ), temelde danışmansız öğrenme olarak kullanılan bir yöntemdir. Amaç görüntülerin boyutu küçülterek daha farklı görüntülerin elde edilmesi için gizli katman adı verilen alanda Gaussian gürültüsü uygulayarak temelde çeşitliliği arttırmaktır. 

Uzaktan algılama, dünyanın yüzeyi hakkında bilgi edinme bilimidir. Günümüzde uzaktan algılama başta askeri hava sahası olmak üzere deprem hazırlığı aşamasında toplanma bölgelerinin de belirlenmesinde önemli rol oynamaktadır. Bu tez çalışmasında uydudan çekilmiş Hiperspektral Görüntüleri (Hyperspectral İmages (HSI) ) CNN modelini kullanarak sınıflandırılması yapılacaktır.
\\
\\
\textbf{Anahtar Kelimeler}: Evrişimsel Sinir Ağları, Varyasyonel Oto-Kodlayıcı, Derin Öğrenme, 

                            \hspace{2.7cm}Uzaktan Algılama

\addcontentsline{toc}{chapter}{ÖZET}