\chapter{Giriş}
Yüksek hesaplamaların daha hızlı yapıldığı günümüz teknolojisinde artık işlem yükü fazla olan projeler daha kolaylıkla yapılmaktadır. Evrişimli sinir ağları (ESA) (Convolution Neural Networks (CNN) ) görüntüler için sınıflandırma problemlerini çözmek amaçlı geliştirilmiş bir derin öğrenme modelidir. Uzaktan algılama, dünyanın yüzeyi hakkında bilgi edinme bilimidir. Günümüzde uzaktan algılama başta askeri hava sahası olmak üzere deprem hazılığı aşamasında toplanma bölgelerinin de belirlenmesinde önemli rol oynamaktadır. Derin öğrenme, öğrenme düzeylerini temsil eden makine öğrenmesinin dalını ifade eder yani burada bahsi geçen derin kelimesi, sınıflandırılacak görüntülerin özelliklerini sinir ağı yapısının derinliğinden alır. Varyasyonel Oto-Kodlayıcı (VOK) (Variational AutoEncoder (VAE) ), temelde yarı-danışmalı öğrenme olarak kullanılan bir yöntemdir. Amaç görüntülerin boyutu küçülterek daha farklı görüntülerin elde edilmesi için gizli katman adı verilen alanda Gaussian gürültüsü uygulayarak temelde çeşitliliği arttırmaktır. Bu tez çalışmasında uydudan çekilmiş Hiperspektral Görüntüleri (Hiperspectral Images (HSI) ) CNN modelini kullanarak sınıflandırılması yapılacaktır.
\section{Alıntı}

\subsection{Deneme}

\begin{table}[!ht]
\centering
    \begin{threeparttable} % <--- new
    \caption{Sonuçların sayısal değeleri}
        \begin{tabular}{|c|c|c|c|}
        \hline
        \textbf{Veriseti Adı} & \textbf{Kappa(\%)} & \textbf{Ortalama Doğruluk} & \textbf{Genel Doğruluk} \\ \hline
        Indian Pines & 0.989    & 99.038           & 99.038         \\ \hline
        Salinas & 0.999    & 99.881          & 99.881         \\ \hline
        \end{tabular}
    \end{threeparttable} % <--- new
\end{table}
Bu bölümde daha sonra güncelleme yapılacak

\begin{table}[!ht]
\centering
\begin{threeparttable} % <--- new
\caption{Word ile Latex'in çeşitli kriterlere göre 3 puan üzerinden değerlendirilmesi}
\label{table:wordvslatex}
\begin{tabular}{|c|c|c|}
\hline
\textbf{Özellik}                                                                               & \textbf{Word Puanı} & \textbf{Latex Puanı} \\ \hline
Küçük doküman hazırlama hızı                                                                   & 3                   & 2                    \\ \hline
\begin{tabular}[c]{@{}c@{}}Büyük doküman hazırlama ve \\ grafiklerle uğraşma hızı\end{tabular} & 1                   & 3                    \\ \hline
Kullanma kolaylığı                                                                             & 3                   & 1                    \\ \hline
Düzen ve çıktı kalitesi                                                                        & 2                   & 3                    \\ \hline
Bilimsel özellikler                                                                            & 1                   & 3                    \\ \hline
Ücret ve kullanılabilirlik (erişilebilirlik)                                                   & 1                   & 3                    \\ \hline
Uyumluluk                                                                                      & 2                   & 2                    \\ \hline
\textbf{Toplam}                                                                                & \textbf{13}         & \textbf{18}          \\ \hline
\end{tabular}
\end{threeparttable} % <--- new
\end{table}
\newpage
\subsection{Alt Bölüm}
\begin{figure}[!ht]
  \centering
  \includegraphics[width=0.6\textwidth]{Figures/ktuLogo}
  \caption{KTU Logo }
  \label{fig:wordvslatex}
\end{figure}

\subsubsection{Alt Alt Bölüm}