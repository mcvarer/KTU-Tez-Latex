\chapter{Giriş}
\thispagestyle{empty}

Yüksek hesaplamaların daha hızlı yapıldığı günümüz teknolojisinde artık işlem yükü fazla olan projeler daha kolaylıkla yapılmaktadır. Evrişimli sinir ağları (ESA) (Convolution Neural Networks (CNN) ) görüntüler için sınıflandırma problemlerini çözmek amaçlı geliştirilmiş bir derin öğrenme modelidir. Uzaktan algılama, dünyanın yüzeyi hakkında bilgi edinme bilimidir. Günümüzde uzaktan algılama başta askeri hava sahası olmak üzere deprem hazılığı aşamasında toplanma bölgelerinin de belirlenmesinde önemli rol oynamaktadır. Derin öğrenme, öğrenme düzeylerini temsil eden makine öğrenmesinin dalını ifade eder yani burada bahsi geçen derin kelimesi, sınıflandırılacak görüntülerin özelliklerini sinir ağı yapısının derinliğinden alır. Varyasyonel Oto-Kodlayıcı (VOK) (Variational AutoEncoder (VAE) ), temelde yarı-danışmalı öğrenme olarak kullanılan bir yöntemdir. Amaç görüntülerin boyutu küçülterek daha farklı görüntülerin elde edilmesi için gizli katman adı verilen alanda Gaussian gürültüsü uygulayarak temelde çeşitliliği arttırmaktır. Bu tez çalışmasında uydudan çekilmiş Hiperspektral Görüntüleri (Hiperspectral Images (HSI) ) CNN modelini kullanarak sınıflandırılması yapılacaktır.
\section{Hiperspektral Nedir}
Hiperspektral görüntüleme yöntemleri askeri uygulamar, kimyasal analizler, tarım alanı analizeleri gibi birçok alanda kullanılmaktadır. Görüntünün kaydedilmesi esnasında her piksel değerine objeden yansıyan ışığın dalga boyları kaydedilerek oluşturulan matris verileri ile hipersipektral görüntü elde edilmektedir. Hiperspektral görüntüleme ile görünür dalga boyu (400-700nm) haricinde kızıl ötesi gibi diğer dalga boylarına ait veriler de elde edilmektedir. Bu veriler yardımı ile spektral imzalar elde edilmektedir. Hiperspektral görüntüleme uygulamaları çoğunlukla görünür ışıkta aynı renge sahip birbirine karışmış birçok bileşenin ayırt edilmesinde kullanılmaktadır. Bu çalışamda genel kullanınım aksine, hiperspektral görüntüleme yöntemi nesnelerin spektral imzasının zamanla değişimini konu almaktadır. 
\subsection{Hiperspektral ile Gabor}
Son yıllarda hiperspektral görüntülerin sınıflandırılmasında
da Gabor filtrelerinden faydalanılmıştır. Bu
çalışmalarda Gabor filtrelerinin kullanım amacı genel olarak
piksellerin uzaysal, spektral ve hibrid uzaysal-spektral
karakteristiklerinin elde edilmesi olarak vurgulanmıştır.
Gabor filtreleri içerdikleri farklı ölçek ve açılarda frekans
bileşenleri ile yukarıda sayılan piksel bilgilerini
sağlayabilmektedir. Gabor filtre bankası kısa dönemli
Fourier dönüşümünün geliştirilmiş hali olduğu için
incelenen görüntülerde Fourier dönüşümünün avantajlarını
sunmaktadır. Hiperspektral görüntülerdeki bazı ayırt
edilmesi zor piksel bölgelerinin tanınmasında bu bilgiler
katkı sağlamıştır.
\begin{table}[!ht]
\centering
    \begin{threeparttable} % <--- new
    \caption{Sonuçların sayısal değeleri}
        \begin{tabular}{|c|c|c|c|}
        \hline
        \textbf{Veriseti Adı} & \textbf{Kappa(\%)} & \textbf{Ortalama Doğruluk} & \textbf{Genel Doğruluk} \\ \hline
        Indian Pines & 0.989    & 99.038           & 99.038         \\ \hline
        Salinas & 0.999    & 99.881          & 99.881         \\ \hline
        \end{tabular}
    \end{threeparttable} % <--- new
\end{table}

\subsection{Matematik Formülleri}
Metin içinde formül örneği $f(x) = x^2$ .
\begin{equation}
  1 + 2 = 3 
\end{equation}

\begin{align}
  f(x) &= x^2
\end{align}

\begin{align}
          \sin A \cos B &= \frac{1}{2}\left[ \sin(A-B)+\sin(A+B) \right] \\
          \sin A \sin B &= \frac{1}{2}\left[ \sin(A-B)-\cos(A+B) \right] \\
          \cos A \cos B &= \frac{1}{2}\left[ \cos(A-B)+\cos(A+B) \right] 
\end{align}

\begin{equation} 
    \label{eq1}
    \begin{split}
        A & = \frac{\pi r^2}{2} \\
         & = \frac{1}{2} \pi r^2
    \end{split}
\end{equation}

\begin{align}
\left[
\begin{matrix}
1 & 0\\
0 & 1
\end{matrix}
\right]
\end{align}

\begin{align}
x(t) \circledast h(t) &= y(t) \\
X(f) H(f) &= Y(f) 
\end{align}

\subsection{Derin öğrenme ile Hiperspektral}
Derin öğrenme algoritmaları arasında en sık kullanılanı
Evrişimsel Sinir Ağlarıdır (ESA). ESA mimarisi birçok
katmandan oluşan bir mimariye sahip olup klasik Yapay
Sinir Ağlarının (YSA) bir türüdür. Bu katmanlar ele alınan
probleme göre sayıları değişebilen evrişim (konvolüsyon)
katmanları, havuzlama (pooling) katmanları, doğrultulmuş
doğrusal ünite (ReLU) ve tam bağlı (fully-connected)
katmanlardır. Mevcut ESA mimarileri herhangi bir
düzenleme yapılmadan doğrudan hiperspektral görüntü
sınıflandırma problemine uygulandıklarında başarı oranı
istenen düzeylerde olmamaktadır. Özellikle benzer renk ve
dokusal içerik bulunan sınıflar için evrişimsel özelliklere ek
olarak daha farklı özelliklerin de elde edilmesi
gerekmektedir.
\begin{figure}[!h]
  \centering
  \begin{measuredfigure} % figure left side
  \includegraphics[width=0.3\textwidth]{Figures/ktuLogo}
  \caption{KTU Logo }
  \label{fig:wordvslatex}
  \end{measuredfigure}
\end{figure}

\subsubsection{ESA Mimari özet}
 Ayrıca ESA mimarileri çok büyük ölçekte
eğitim örneğine ihtiyaç duyarken, hiperspektral görüntü veri
tabanlarının sınırlı sayıda örnek içermesi derin öğrenme
mimarisinin eğitilmesinde yetersiz kalınmasına yol
açmaktadır. Bu çalışmada bahsedilen zorlukların üstesinden
gelebilecek bir mimariye sahip olmakla birlikte, özellik
çıkarma ve öğrenme kapasitesi yüksek bir ESA mimarisi
tasarlanmıştır. 