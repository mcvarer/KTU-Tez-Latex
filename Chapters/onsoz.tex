\setstretch{1.5}

\chapter*{ÖNSÖZ}

\hspace{0.8cm}  Bu çalışmada plevral efüzyon sitopatolojik görüntülerin derin öğrenme yöntemiyle
otomatik algılanması sağlanmıştır. Veri tabanı hazırlama sürecinde ise görüntüler mikroskoptan
okunarak panorama yöntemiyle birleştirilmiştir ve bu görüntüden belirli boyutlarda görüntüler
kesilerek hazırlanması sağlanmıştır. Bu çalışmada danışmanlığımı üstlenen değerli hocam Dr. Öğr. Üyesi Murat AYKUT'a ilgi, destek ve tecrübelerinden dolayı teşekkürü bir borç bilirim.
Çalışmam boyunca bana her türlü desteği sağlayan yüksek
lisans eğitimim boyunca sabır, destek ve sevgileriyle yanımda olan aileme ve dostlarıma çok
teşekkür ederim.

\vspace{1.5cm}
\hfill Murat Can VARER

\hfill Trabzon 2021

\addcontentsline{toc}{chapter}{ÖNSÖZ}