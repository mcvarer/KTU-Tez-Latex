\chapter{Genel Bilgiler}

Hiperspektral Görüntüleme (HSG), yüzey materyallerinden yansıyan enerjinin, dar ve bitişik çok sayıda dalga boyu bandında
ölçümüdür. Hiperspektral sensörler kullanılarak elde edilen veriler ilk iki boyut uzamsal üçüncü boyut spektral bilgiyi içeren hacimsel
verilerdir. Ayrıca her bir piksel yüksek boyutlu vektörlerden oluşmaktadır. Elektromanyetik spektrum bandında hem görünür hem de
kızılötesi bölgesinde birçok dalga boyunda görüntüler elde edebilen HSG, daha az sayıda görüntüler elde edebilen multispektral
görüntülemeye göre daha anlamlı özellikler içermektedir. Bu özellikler farklı nesneleri tespit etmede yardımcı olabilmektedir. HSG
kullanılarak nesnelerin tespiti ve sınıflandırılması gibi son zamanlardaki çalışmalar, bağlamsal özelliklerin büyük avantajlar
sağlayabileceğini göstermiştir (Lee ve Kwon, 2016; Chu ve ark., 2018).\\

Hiperspektral görüntülerin sınıflandırılmasında en yakın
komşu, karar ağaçları, destek vektör makinesi (DVM) ve
yapay sinir ağları (YSA) gibi farklı makine öğrenmesi
temelli metotlar kullanılmaktadır. En yakın komşu metodu
Öklid uzaklığı kullanan sınıflandırılma yöntemidir. \cite{blanzieri2008nearest}. DVM
çekirdek metodu kullanarak yüksek boyutlu uzayda farklı
hiperspektral sınıflar arasında sınıflandırma sınırı
belirlenmektedir \cite{melgani2004classification}. DVM kadar yüksek sınıflandırma
doğruluğu elde edememelerine rağmen, YSA kullanılarak
yapılan hiperspektral görüntü sınıflandırma çalışmaları da
bulunmaktadır \cite{ratle2010semisupervised}. Bu tür geleneksel spektral
sınıflandırıcıların çoğu hiperspektral görüntülerdeki bazı
bölgelerin tespit edilip \hfill ayırt  edilmesinde halen yetersiz
kalabilmektedir. Bunun temel nedeni sınırlı sayıda
etiketlenmiş hiperspektral görüntü verisinin bulunmasıdır
\cite{kang2014intrinsic}. \\
Derin öğrenme metotlarından birisi olan Evrişimli Sinir Ağı (ESA) nesne algılama (Ren ve ark., 2015), görüntü sınıflandırma
(Wang ve ark., 2016), derinlik tahmini (Liu ve ark., 2015), anlamsal bölümleme (Gidaris ve Komodakis, 2015), cilt kanseri
sınıflandırması (Saba ve ark., 2019) gibi alanlarda yüksek başarımlara sahiptir. Bunun sebebi, ESA’nın çok fazla ön işleme olmadan
ağda bulunan gizli katmanlar ile özellikleri çıkarabilmesidir. \\
\newpage
Hiperspektral görüntü sınıflandırmada karşılaşılan bir diğer zorluk ise farklı maddelerin benzer spektral değerlere sahip
olması durumudur. Bu durumda sadece spektral bilgi ile
sınıflandırma yapmak zordur. Bu problemi çözmek için
uzaysal ve spektral bilgileri birlikte kullanan markov rastgele
alanlar (MRA) yöntemi kullanılmıştır \cite{tarabalka2010svm} \\

Yukarıda bahsedilen metotlar görüntü özelliklerini
sınıflandırmadan bağımsız bir dizi işlem ile
çıkartmaktadırlar. Ayrıca gerekli durumlarda uzman
deneyiminden faydalanılmakta olup parametre ayarlaması
gibi bazı işlemlere de ihtiyaç duyabilmektedirler. Derin
öğrenme metotları daha dinamik ve yüksek seviyeli görüntü
özellikleri sunarak hiperspektral görüntülerin
sınıflandırılmasında yoğun bir şekilde kullanılmaktadırlar
\cite{li2017hyperspectral, yu2017convolutional}. Derin sinir ağları etkin ve adaptif öğrenme
modelleridir. Çok katmanlı yığılmış oto-kodlayıcı, derin
Boltzmann makineleri ve evrişimsel sinir ağları yaygın bir
şekilde kullanılan derin sinir ağları arasında yer almaktadır.
Evrişimsel sinir ağları (ESA) iki boyutlu sinir ağı olup hem
uzaysal hem de spektral bilgiyi daha iyi
yakalayabilmektedir. Chen vd. \cite{chen2014deep} TBA, oto-kodlayıcı ve
lojistik regresyonu birlikte kullanarak hiperspektral görüntü
sınıflandırması yapan bir yöntem geliştirmişlerdir. ESA
kullanan bir diğer çalışmada ise beş katmanlı bir ESA
mimarisi inşa edilerek hiperspektral görüntüleri optimum
şekilde analiz edebilecek bir ağ tasarlanmıştır \cite{hu2015deep}. Zabalza
vd. \cite{zabalza2016novel} yığılmış oto-kodlayıcı kullanarak hiperspektral
görüntüyü farklı bölgelere bölütlemişlerdir. Derin öğrenme
modellerinin sahip olduğu ağır parametre yükü\\ hafifletilerek
geliştirilen bir ESA mimarisinde ise hiperspektral
görüntülemedeki komşu piksellerin iç korelasyon
bilgilerinden faydalanılmıştır \cite{li2016hyperspectral}. \\

Gözetimsiz bir ağ modeli olan Varyasyonel Otokodlayıcı (Variational Autoencoder (VAE) ), klasik otokodlayıcı
modelinde olduğu gibi kodlayıcı (E) ve kod çözücü (G)
ağlarından oluşan bir ağ modelidir. Klasik otokodlayıcı
modelinde karşılaşılan gradyan düşüşü ve aşırı öğrenme
problemlerini çözmek üzere otokodlayıcıya bir varyasyon
terimi eklenmesi ile elde edilmiştir \cite{kingma2013auto}.\\
\newpage
HSG verilerinin sınıflandırılmasında 2B-ESA kullanıldığında sadece uzamsal özellikler elde edilir. HSG verileri 3 boyutlu
hacimsel veriler olduğu için hem uzamsal hem de spektral özelliklerin elde edilmesi gerekir (Roy ve ark., 2019). Bu özellikler 3
boyutlu konvolüsyon katmanları kullanılarak sağlanabilir. Ayrıca hiperspektral veriler için komşu pikseller büyük önem taşımaktadır.
Bu nedenlerden dolayı makalede hem uzamsal hem de spektral özelliklerin elde edilmesi için 3B-ESA kullanılmıştır. Ayrıca HSG
sınıflandırılmasında yeni bir yöntem olan komşuluk çıkarımı yöntemi kullanılarak komşu pikselleride içerecek şekilde mini küpler
oluşturulmuştur. Bu sayede daha çok örnek kullanılarak sınıflandırılma performansının artması sağlanmıştır. Sınıflandırma
performansını değerlendirmek için Indian Pines (IP), Salinas scene (SA) gibi iki uzaktan algılama veriseti
kullanılmıştır. Sınıflandırma işlemi sonucunda önderilen 3B-ESA-VAE modeli yüksek başarımlar elde etmiştir.
% \section{genel2}